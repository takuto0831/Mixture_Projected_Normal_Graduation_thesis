\documentclass[a4j,11pt]{jarticle}
%\usepackage[dviout]{graphicx}
\usepackage[dvipdfmx]{graphicx}
\usepackage{amsmath}
\usepackage{amssymb}
\usepackage{ascmac}
%\usepackage{epsbox}
\usepackage{float}
\usepackage{here}
\usepackage{lscape}
\usepackage{latexsym}
\usepackage{pifont}
\usepackage{wrapfig}
\usepackage{type1cm}
\usepackage{algorithm}
\usepackage{algorithmic}
\usepackage{txfonts}
\usepackage{bm}
\usepackage{comment}
\usepackage{url}
%\usepackage{natbib}
%\usepackage[square]{natbib}

\usepackage{listings}
%\usepackage{plistings}

%\setlength{\voffset}{-25.4mm}
\setlength{\topmargin}{-17.5mm}   %トップとヘッダの間隔
%\setlength{\headheight}{20mm}   %ヘッダの高さ
%\setlength{\headsep}{0mm}   %ヘッダとテキストの間隔
\setlength{\textwidth}{45zw}   %テキストの幅
\setlength{\hoffset}{-10mm}
\setlength{\textheight}{45\baselineskip}   %テキストの高さ
%\addtolength{\textheight}{\topskip}
%\setlength{\footskip}{0mm}
%\setlength{\oddsidemargin}{21.5mm}   %サイドとテキストの間隔(奇数ページ)
%\setlength{\evensidemargin}{21.5mm}   %サイドとテキストの間隔(偶数ページ)
\pagestyle{empty}   %ページ番号なし
\newcommand{\g}[1]{\boldsymbol{#1}}
\newcommand{\lw}[1]{\smash{\lower2.0ex\hbox{#1}}}
\renewcommand{\baselinestretch}{1.0}

\makeatletter
\def\theequation{\arabic{equation}}   %数式番号を(章.式)形式
\@addtoreset{equation}{section}
%\def\thefigure{\thesection.\arabic{figure}}   %図番号を(章.図)形式
%\@addtoreset{figure}{section}
%\def\thetable{\thesection.\arabic{table}}   %表番号を(章.表)形式
%\@addtoreset{table}{section}
\def\tr{\mathop{\operator@font tr}\nolimits}
\def\grad{\mathop{\operator@font grad}\nolimits}
\def\St{\mathop{\operator@font St}\nolimits}
\def\Hess{\mathop{\operator@font Hess}\nolimits}
\def\D{\mathop{\operator@font D}\nolimits}
\def\sym{\mathop{\operator@font sym}\nolimits}
\def\s.t.{\mathop{\operator@font s.t.}\nolimits}
\def\diag{\mathop{\operator@font diag}\nolimits}
\def\section{\@startsection{section}{1}{\z@}
   {0.8\Cvs \@plus.5\Cdp \@minus.2\Cdp}
   {0.2\Cvs \@plus.3\Cdp}
   {\normalfont \Large \bfseries}}
\makeatother
\makeatletter
\def\subsection{\@startsection{subsection}{1}{\z@}
   {0.8\Cvs \@plus.5\Cdp \@minus.2\Cdp}
   {0.2\Cvs \@plus.3\Cdp}
   {\normalfont \normalsize \bfseries}}
\makeatother
\makeatletter
\newcommand{\figcaption}[1]{\def\@captype{figure}\caption{#1}}
\newcommand{\tblcaption}[1]{\def\@captype{table}\caption{#1}}
\makeatother

\begin{document}
%\bibliographystyle{jecon}
%\bibliographystyle{apalike} %いらない

\begin{center}
{\Large \textbf{混合 Projected Normal 分布によるクラスタリングの性能評価}}
\end{center}
\begin{flushright}
小坪 琢人(塩濱 敬之准教授)
\end{flushright}
\vspace{-3zh}

%%%%%%%%%%%%% これ以下, 本文 %%%%%%%%%%%5%%
%%%%%%%%%%%%%  section1  はじめに %%%%%%%%%%%%%%%

\section{はじめに}
\vspace{-0.5zh}

Projected Distribution は平面または空間上の放射状の射影によって得られる. より一般的には, 多変量正規ベクトルをノルムで割ることで, 単位超球面上への射影分布が得られる. 多変量正規ベクトル($k$次元)を$X$として, $k \geq 2$ の場合には, 単位超球面上への単位ベクトル $U$ は $U = X/||X||$ で表される. このとき$U$は$k$次元の General Projected Normal Distributionに従い, $U \sim \mathcal{PN}_k(\mu,\Sigma)$と表せる. General Projected Normal Distribution は, パラメータ$\mu, \Sigma$をもち, Projected Normal Distribution では$\Sigma = \mathcal{I}$と定義されていた確率分布を一般化したものである\cite{PML}. $\mathcal{PN}_k(\mu,\mathcal{I})$は平均方向$\mu$に対して, 単峰性かつ対称の分布となるが, $\mathcal{PN}_k(\mu,\Sigma)$では対称分布もしくは二峰性分布となる\cite{PN1}.


%%%%%%%%%%%%%%%%%%%%%%%%%%%%%%%%%%%%%%%%%%%%%%%%%%%%%%%%%%%%%%%%%%%%%%%%%%%%%%%%%%%%%%%%%%%%%%%%%%%%%%%%%%%%%%%
%%%%%%%%%%%%%%%%%%%%%%%% senction2 混合分布 %%%%%%%%%%%%%%%%%%%%%

\section{Mixture Projected Normal Distribution}

\subsection{Projected Normal Distribution}

単位ベクトル$U$は, 座標系の適切な選択によって指向性ランダムベクトル$\Theta$として解釈される. 円形の場合, ランダム方向$\Theta$は極座標上の$U = (\cos\Theta, \sin\Theta)^T$から得られる.

Wang and Gelfand (2013) and Wang and Gelfand (2014) によると,$\Sigma \neq I$のもとで$\mathcal{PN}_2(\mu,\Sigma$)のとき, 特別な円形データの場合, $U = (\cos\Theta, \sin\Theta)^T$の密度は以下で示せる. \cite{SKM},\cite{MixtureVonMises},\cite{GPN}


\begin{eqnarray}
\label{PNC}
p(\theta|\mu, \Sigma) = \frac{1}{2\pi A(\theta)}|\Sigma|^{-\frac{1}{2}}
\exp(C)\left\{1 + \frac{B(\theta)}{\sqrt{A(\theta)}} \frac{\Phi \left(\frac{B(\theta)}{\sqrt{A(\theta)}}\right)}{\phi \left(\frac{B(\theta)}{\sqrt{A(\theta)}}\right)}\right\} I_{[0,2\pi)}(\theta)
\end{eqnarray}

$u^T = (\cos\theta,\sin\theta), A(\theta) = u^T\Sigma^{-1}u, B(\theta) = u^T \Sigma^{-1} \mu, C = -\frac{1}{2} \mu^T \Sigma^{-1} \mu$であり, $I_{[0,2\pi)} (\cdot)$は指示関数, $\Phi(\cdot),\phi(\cdot)$ は標準正規分布の確率密度関数と累積密度関数である.

%%%%%%%%%%%%%%%%%%%%% projected normal の多次元への拡張についてかく %%%%%%%%%%%%%%%%%%%%%%%%

\subsection{General projected normal distribution}


%%%%%%%%%%%%%%%%%%%%%%%%%%%%%%%%%%%%%%%%%%%%%%%%%%%%%%%%%%%%%%%%%%%
%%%%%%%%%%%%%%%%%%%%%%%  section3 数値実験 %%%%%%%%%%%%%%%%%%%%%%%%

\section{数値実験}

人工的データを用いて, 混合 Projected Normal 分布によるクラスタリングの性能評価を行う.

\subsection{解析データ}


%%%%%%%%%%%%%%%%%%%%%%%%%%%%%%%%%%%%%%%%%%%%%%%%%%%%%%%%%%%%%%%%%%%%%%
%%%%%%%%%%%%%%%%%%%%%%%  section4 まとめ %%%%%%%%%%%%%%%%%%%%%%%%%%%%%

\section{まとめ}


%\newpage
\addcontentsline{toc}{section}{参考文献} %目次に参考文献を入れる

%必要になる
%\newpage
%\section{付録}

%参考文献を引用する際に必要なコマンド
\bibliographystyle{jplain}
\bibliography{bunken}

\end{document}

%\begin{table}[H]
%\begin{center}
%\caption{条件付確率表(CPT)}   %キャプション
%\label{cpt}   %ラベル
%\begin{tabular}{|c||c|c|c|}   %{}で文字の揃え方を指定
%\hline
% & $Pa(X_{j})=x_{1}$ & \dots & $Pa(X_{j})=x_{m}$
%\\ \hline
%$X_j=y_1$ & $p(y_1|Pa(X_j)=x_1)$ & \dots & $p(y_1|Pa(X_j)=x_m)$
%\\ \hline
%$\vdots$ & $\vdots$ & $\ddots$ & $\vdots$
%\\ \hline
%$X_j=y_n$ & $ p(y_n|Pa(X_j)=x_1)$ & \dots & $p(y_n|Pa(X_j)=x_m)$
%\\ \hline
%\end{tabular}
%\end{center}
%\end{table}
