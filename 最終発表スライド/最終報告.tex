\documentclass[dvipdfmx]{beamer} %通常用
%\documentclass[dvipdfm,handout]{beamer} #ハンドアウト作成用

\AtBeginDvi{\special{pdf:tounicode 90ms-RKSJ-UCS2}} % 栞の文字化けを制御(日本語の場合必須)
\setbeamertemplate{navigation symbols}{} %ナビゲーションバーを消す

\usepackage{comment}
\usepackage{amsmath}
\usepackage{algorithm}
\usepackage{algorithmic}

%%% 以下2つはハンドアウト印刷用
%\usepackage{pgfpages}
%\pgfpagesuselayout{4 on 1}[border shrink=3mm]

%%% 付録をページ番号に含めないためのコマンド
\newcommand{\backupbegin}{
\newcounter{framenumberappendix}
\setcounter{framenumberappendix}{\value{framenumber}}
}
\newcommand{\backupend}{
\addtocounter{framenumberappendix}{-\value{framenumber}}
\addtocounter{framenumber}{\value{framenumberappendix}}
}

%%% メインテーマ
%\usetheme{Berkeley}
%\usetheme{CambridgeUS}
%\usetheme{Default}
%\usetheme{Darmstadt}
%\usetheme{Hannover}
%\usetheme{lankton-keynote}
%\usetheme{Luebeck}
%\usetheme{Marburg}
\usetheme{Madrid}
%\usetheme{boxes}
%\usetheme{Bergen}
%\usetheme{Boadilla}
%\usetheme{Pittsburgh}
%\usetheme{Rochester}

%%% テーマ
%\useinnertheme{rectangles}
%\useoutertheme{default}

%%% カラーテーマ(省略可)
%\useoutertheme{infolines}
%\usecolortheme[RGB={64,64,64}]{structure}     
%\definecolor{babyblue}{rgb}{0.54,0.81,0.94}                                                                                                
%\usecolortheme{dolphin}
%\usecolortheme{beaver}
%\usecolortheme{beetle}
%\usecolortheme{crane}
%\usecolortheme{dolphin}
%\usecolortheme{seagull}
%\usecolortheme{wolverine}
%\usecolortheme{spruce}
%\usecolortheme{rose}
%\usecolortheme{seahorse}
\setbeamertemplate{footline}[page number]

%%% フォント
\renewcommand{\kanjifamilydefault}{\gtdefault} % 日本語フォントをゴシック
\usefonttheme[onlymath]{serif}
\usefonttheme[onlylarge]{structurebold}
%\usefonttheme{professionalfonts}
\fontencoding{\encodingdefault}
\fontfamily{\kanjifamilydefault}
\fontseries{\seriesdefault}
\fontshape{\shapedefault}
\selectfont
%\mathversion{bold} % 数式フォントをbold体

%%% インナー, アウターテーマ(省略可)
%\useinnertheme{circles}
%\useoutertheme{infolines}

%\logo{\includegraphics[width=1.5cm, height=1.5cm]{.jpg}} % ロゴをいれる
\setbeamertemplate{navigation symbols}{} % ナビゲーションバーなし
%\setbeamertemplate{background}[grid][step=5mm] % 背景グリッド
\setbeamertemplate{footline}[frame number] % ページ番号の表示
\setbeamerfont{footline}{size=\small,series=\bfseries}
\setbeamercolor{footline}{fg=black,bg=black}
\setbeamertemplate{caption}[numbered] % 図表番号の表示
%\setbeamerfont*{frametitle}{size=\normalsize,series=\bfseries} % フレーム文字の大きさ
\setbeamerfont*{frametitle}{size=\large,series=\bfseries} % フレームごとのフォントを設定変更できる。
\setbeamertemplate{frametitle}[default][center] % タイトルを中央寄せに設定変更できる。

\definecolor {mycolor1} {rgb} {0.00, 0.39, 0.00}
\definecolor {mycolor2} {rgb} {0.55, 0.27, 0.07}
\definecolor {mycolor3} {rgb} {0.63, 0.13, 0.94}

\definecolor {mycolorTitle} {rgb} {0.85, 0.855, 0.85}
\definecolor {mycolorHeader} {rgb} {0.93, 0.935, 0.93}

%ヘッダーとタイトルの色(fgで文字の色変えられる)
%\setbeamercolor{frametitle}{bg = mycolorHeader}
%\setbeamercolor{title}{bg = mycolorTitle}

\def\conpage{7}

%%% パッケージ
\usepackage[japanese]{babel}
\usepackage{inputenc}
\usepackage{times}
\usepackage{amsmath}
\usepackage{amssymb}
\usepackage{amsfonts}
\usepackage[T1]{fontenc}
\usepackage{hyperref}
\usepackage{algorithm,algorithmic}
\usepackage{ascmac}
%\usepackage{txfonts}
\usepackage{color}
%\usepackage{algpseudocode,algorithm}
%\usepackage{tikz}
%\usetikzlibrary{arrows}
%\tikzstyle{block}=[fill=blue,draw opacity=0.7,line width=1.4cm]

%\makeatletter
%\renewcommand{\thealgorithm}{%
%\thesection.\arabic{algorithm}}
%\@addtoreset{algorithm}{section}
%\makeatother

%\usepackage{listings,jlisting}
\usepackage{listings}

\lstset{%
  language={R},
  basicstyle={\small},%
  identifierstyle={\small},%
  commentstyle={\small\itshape},%
  keywordstyle={\small\bfseries},%
  ndkeywordstyle={\small},%
  stringstyle={\small\ttfamily},
  frame={tb},
  breaklines=true,
  columns=[l]{fullflexible},%
  numbers=left,%
  xrightmargin=0zw,%
  xleftmargin=3zw,%
  numberstyle={\scriptsize},%
  stepnumber=1,
  numbersep=1zw,%
  lineskip=-0.5ex%
}

\newcommand{\bm}[1]{\mbox{\boldmath $#1$}}
\newcommand{\mapright}[1]{\mathop{\longrightarrow}\limits_{#1}}
\newcommand{\argmax}{\mathop{\rm argmax}\limits}

\renewcommand{\figurename}{図}
\renewcommand{\tablename}{表}

%%% Title, Author, etc.
\title[タイトル]{混合射影正規分布によるクラスタリングについて}
%\subtitle[サブタイトル]{}
\author[発表者名]{塩濱研究室\\ 小坪琢人}
\institute[所属]{東京理科大学\ 工学部経営工学科4年\\学籍番号 4414036}
\date[日付]{2018年1月30日}

\begin{document}

\begin{frame}[plain]
\titlepage
\end{frame}

\begin{frame}{目次}
\tableofcontents
\end{frame}

%%%%%%%%%%%%% はじめに %%%%%%%%%%%%%%%
\section{はじめに}
\begin{frame}{はじめに}

\begin{itemize}

\item 
円周上や球面上のデータを扱う統計手法を方向統計学といい, 近年多様体上の統計分析
手法として, 注目を集めている.

\item 
単位超球面上$(\mathbb{S}^{p-1})$ に分布するようなデータをユークリッド空間上のデータとして扱い, ベクトル間の類似度の指標にユークリッド距離を用いると良い解析が行えない場合がある.

\end{itemize}

\end{frame}

\begin{frame}{背景}

\begin{itemize}

\item 
Dhillon and Modha (2001) は, ユークリッド距離に基づく非類似度の尺度を単位球面上に
射影したコサイン非類似度の最小化に基づく超球面上の$k$平均法を提案した.

\item  
超球面上の$k$平均法は確率モデルを仮定しないノンパラメトリックな手法であるのに対
し, パラメトリックな超球面上のクラスタリング手法として, Gopal and Yang (2014) によ
るvon Mises Fisher 分布の混合分布を用いた手法がある.

\end{itemize}

\end{frame}

%%%%%%%%%%%%% 目的 %%%%%%%%%%%%%%%
\section{目的}
\begin{frame}{目的}
\begin{block}{目的}
\begin{itemize}

\item
方向データの分布として知られる, 射影正規分布の混合分布によるクラスタリングの性能評価を行う.

\end{itemize}
\end{block}
\end{frame}

%%%%%%%%%%%%% 混合射影正規分布 %%%%%%%%%%%%%%%
\section{混合射影正規分布}
\begin{frame}{射影正規分布(1/3)}

Wang and Gelfand (2013)によると, $\mathcal{PN}_2(\bm \mu,\Sigma$)の円形データの場合, 単位円上の方向を表す$U = (\cos\Theta, \sin\Theta)^T$における$\theta$の確率密度を式(\ref{PNC})に示す.

\vspace{-0.5cm}
\footnotesize
\begin{eqnarray}
\label{PNC}
p(\theta; \bm \mu, \Sigma) = \frac{1}{2\pi A(\theta)}|\Sigma|^{-\frac{1}{2}}
\exp(C)\left\{1 + \frac{B(\theta)}{\sqrt{A(\theta)}} \frac{\Phi \left(\frac{B(\theta)}{\sqrt{A(\theta)}}\right)}{\phi \left(\frac{B(\theta)}{\sqrt{A(\theta)}}\right)}\right\} I_{[0,2\pi)}(\theta),
\end{eqnarray}
\normalsize

%\noindent
ここで, $\bm u^T = (\cos\theta,\sin\theta), \ A(\theta) = \bm u^T\Sigma^{-1}\bm u, \ B(\theta) = \bm u^T \Sigma^{-1} \bm \mu,$
$\ C = -\frac{1}{2} \bm \mu^T \Sigma^{-1} \bm \mu$であり, $I_{[0,2\pi)} (\cdot)$は指示関数, $\Phi(\cdot), \phi(\cdot)$ は標準正規分布の確率密度関数と累積密度関数である.

\end{frame}
\begin{frame}{射影正規分布(2/3)}
\begin{itemize}
	\item 平均方向の定義
\end{itemize}

\begin{figure}[bp]
 \begin{tabular}{c}
 \begin{minipage}{0.5\hsize}
  \begin{center}
   \includegraphics[clip,height= 40mm]{data/sample_False.png}
\caption{算術平均による平均の定義}
\label{sample_mu1}
  \end{center}
 \end{minipage}
 \begin{minipage}{0.5\hsize}
  \begin{center}
 \includegraphics[clip,height= 40mm]{data/sample_True.png}
\caption{円周上における平均方向の定義}
\label{sample_mu2}
  \end{center}
 \end{minipage}
\end{tabular}
\label{sample_mu}
\end{figure}
\end{frame}

\begin{frame}{射影正規分布(3/3)}
Hernandez-Stumpfhauser et al. (2017) によると, $\mathcal{PN}_3(\bm \mu,\Sigma$)の球形データの場合, 単位球面上の方向を表す$U = (\cos\Theta_1 \sin \Theta_2, \sin\Theta_1 \sin \Theta_2, \cos \Theta_2)^T$における$\bm \theta = (\theta_1, \theta_2)^T$の確率密度を式(\ref{PNS})に示す.

\vspace{-0.5cm}
\footnotesize %式を小さくする
\begin{eqnarray}
\label{PNS}
p(\bm \theta; \bm \mu, \Sigma) &=& \left(\frac{1}{2\pi A(\bm \theta)}\right)^{\frac{3}{2}} |\Sigma|^{-\frac{1}{2}}
\exp(C) \nonumber \\ 
&& \hspace{-1.5cm} \times \left( \left[1 + D(\bm \theta) \frac{\Phi \{D(\bm \theta)\}}{\phi \{D(\bm \theta)\}} \right] D(\bm \theta) + \frac{\Phi \{D(\bm \theta)\}}{\phi \{D(\bm \theta)\}} \right) I_{[0,2\pi)}(\theta_1) I_{[0,\pi)}(\theta_2),
\end{eqnarray}
\normalsize

%\noindent
ここで, $\bm u^T = (\cos\theta_1 \sin \theta_2, \sin\theta_1 \sin \theta_2, \cos \theta_2), \ D(\bm \theta) = B(\bm \theta) A^{-\frac{1}{2}}(\bm \theta),$
$A(\bm \theta) = \bm u^T \Sigma^{-1} \bm u,\ B(\bm \theta) = \bm u^T \Sigma^{-1} \bm \mu, \ C = -\frac{1}{2} \bm \mu^T \Sigma^{-1} \bm \mu$であり, 

$I_{[0,2\pi)} (\cdot), I_{[0,\pi)}(\cdot)$は指示関数, $\Phi(\cdot), \phi(\cdot)$ は標準正規分布の確率密度関数と累積密度関数である. 
\end{frame}

\begin{frame}{混合射影正規分布(1/2)}

$m$個のコンポーネントからなる球面上の射影正規分布の混合分布を式(\ref{MPNS})に示す. 

\vspace{-0.5cm}
\begin{eqnarray}
\label{MPNS}
p(\bm \theta;\bm w,\bm \mu, \Sigma) = \sum^m_{j=1} w_j \mathcal{PN}_3(\bm \theta;\bm \mu_j, \Sigma_j),
\end{eqnarray}

\noindent
ただし, $w_j$は混合比率であり, $0 < w_j < 1$, $\sum^m_{j=1} w_j = 1$を満たす. 混合射影正規分布におけるパラメータは, $w, \bm \mu_j, \Sigma_j$であるが, 球面上の混合射影正規分布のパラメータについては識別性を考慮して分散共分散行列を定式化する必要がある. 分散共分散行列を式(\ref{SIGMA})に示す.

\vspace{-0.5cm}
\begin{eqnarray}
\label{SIGMA}
 \Sigma_j = \left(
    \begin{array}{cc}
      \Sigma^*_j + \bm \gamma_j \bm \gamma_j^T & \bm \gamma_j \\
      \bm \gamma_j^T & 1
    \end{array}
  \right),
\end{eqnarray}
\noindent
ここで, $\Sigma^*_j$は$(2 \times 2)$の正定値対称行列, $\bm \gamma_j$は$(2 \times 1)$のベクトルである. パラメータベクトルは$\bm \eta = (w_1, \dots, w_m, \bm \mu_1, \dots, \bm \mu_m, \Sigma^*_1, \dots, \Sigma^*_m, \bm \gamma_1, \dots, \bm \gamma_m)^T$となる.

\end{frame}

\begin{frame}{混合射影正規分布(2/2)}
角度データ $\bm \theta$ が得られたときの, パラメータベクトル $\bm \eta$の事後分布を $p(\bm \eta| \bm \theta)$, パラメータベクトルの事前分布を $p(\bm \eta)$とすると事後分布は式(\ref{BAYES})で表せる. 

\begin{eqnarray}
\label{BAYES}
p(\bm \eta | \bm \theta) = \frac{p(\bm \theta | \bm \eta) p(\bm \eta)}{p(\bm \theta)} \propto p(\bm \theta | \bm \eta) p(\bm \eta)
\end{eqnarray}

\noindent
すなわち, 事後分布$p(\bm \eta | \bm \theta)$は尤度$p(\bm \theta | \bm \eta)$と事前分布 $p(\bm \eta| \bm \theta)$の積に比例する. ここで, $p(\bm \theta)$はすでに得られた, データ$\bm \theta$にのみ依存する定数値であるので, $p(\bm \theta | \bm \eta) p(\bm \eta)$が $\bm \eta$  の事後分布の核関数を形成し, $p(\bm \theta)$は正規化定数とみなすことができる. 尤度と事前分布の計算は簡単であるが, 周辺分布 $p(\bm \theta)$の計算は一般に簡単ではないので, 事後分布に比例する分布 $p(\bm \theta | \bm \eta) p(\bm \eta)$ から乱数サンプルを発生させる. この方法で得られたサンプルをMCMCサンプルと呼ぶ. 球面上における混合射影正規分布の尤度関数を式(\ref{logPNS})に示す. 式中の$A(\bm \theta), C, D(\bm \theta)$は式(\ref{PNS})の下に示してある.

\begin{eqnarray}
\label{logPNS}
\log p(\bm \theta | \bm \eta) &=& \sum^m_{j=1} \{\log w_j + \log \mathcal{PN}_3(\bm \theta;\bm \mu_j, \Sigma_j)\} \nonumber \\
&=& \sum^m_{j=1} \left[ \log w_j - \frac{3}{2} \log 2\pi - \frac{3}{2} \log A(\bm \theta) - \frac{1}{2} \log |\Sigma_j| + C \right. \nonumber \\
&&\  \left. + \log \left( \left[1 + D(\bm \theta) \frac{\Phi \{D(\bm \theta)\}}{\phi \{D(\bm \theta)\}} \right] D(\bm \theta) + \frac{\Phi \{D(\bm \theta)\}}{\phi \{D(\bm \theta)\}} \right)\right] \nonumber \\
&\propto& \sum^m_{j=1} \left[ \log w_j - \frac{3}{2} \log A(\bm \theta) - \frac{1}{2} \log |\Sigma_j| + C \right. \nonumber \\
&&\ \left. + \log \left( \left[1 + D(\bm \theta) \frac{\Phi \{D(\bm \theta)\}}{\phi \{D(\bm \theta)\}} \right] D(\bm \theta) + \frac{\Phi \{D(\bm \theta)\}}{\phi \{D(\bm \theta)\}} \right) \right], 
\end{eqnarray}

\end{frame}

%%%%%%%%%%% 解析手法 %%%%%%%%%%%%%
\section{解析手法}
\begin{frame}{t-SNE}
\begin{eqnarray}
\label{tsne1}
P_{j|i} = \frac{\exp(-\|x_i - x_j\|^2 / 2\sigma_i^2}{\Sigma_{k \neq i}\exp(-\|x_i - x_k\|^2/ 2\sigma_i^2)},\ 
P_{ij} = \frac{P_{j|i} + P_{i|j}}{2N},
\end{eqnarray}

\begin{eqnarray}
\label{tsne2}
Q_{ij} = \frac{(1 + \|y_i - y_j\|^2)^{-1}}{\Sigma_{k \neq i}(1 + \|y_i - y_k\|^2)^{-1}},
\end{eqnarray}

ここでは, 分布間の近さを測る方法として, KL(カルバック$\cdot$ライブラー)情報量を用いる. 圧縮前の確率分布と圧縮後の確率分布のKL
情報量を損失関数$L$として, $L$が最小になるような$y_i$を求めることで, 低次元データ$\bm y$を取得する. 損失関数$L$を式$(\ref{tsne3})$に示す.

\begin{eqnarray}
\label{tsne3}
L = KL(P || Q) = \Sigma_i \Sigma_j P_{ij} \log \frac{P_{ij}}{Q_{ij}}.
\end{eqnarray}
\end{frame}

\begin{frame}{HMC}
\begin{enumerate}

\item{}
$\eta_j(0), q_j(0)$の初期値を設定する. $\eta_j(0)$は各パラメータの事前分布からの設定し, $q_j(0)$は$N(0,M_j)$からランダムサンプリングする. $M_j$ は任意の分散パラメータである. 
 
\item{}

以下の式に従い, $\eta_j, \bm q_j$を更新する. $\epsilon$は状態変化のステップ幅を表す.

\begin{eqnarray*}
&&q_j(t+\frac{\epsilon}{2}) = q_j(t) - \frac{\epsilon}{2} \frac{\partial U}{\partial \eta_j} (\bm \eta(t)), \\
&&\eta_j(t+\epsilon) = \eta_j(t) + \epsilon \frac{q_j(t + \frac{\epsilon}{2})}{M_j}, \\
&&q_j(t+\epsilon) = q_j(t+\frac{\epsilon}{2}) - \frac{\epsilon}{2} \frac{\partial U}{\partial \eta_j} (\bm \eta(t + \epsilon)),
\end{eqnarray*}

\item{}
得られたパラメータの採択, 棄却を決定する. 上記のステップで得られたパラメータを$\eta^*_j, q^*_j$として,
現在が$t$回目の反復とすると, サンプルの棄却率は以下の式で定義する.

\begin{eqnarray*}
r_j = \frac{p(\eta^*_j|\theta_j) p(q^*_j)}{p(\eta_j(t-1)|\theta_j) p(q_j(t-1))}.
\end{eqnarray*}

\item{}
一様乱数 $u_j \sim U(0,1)$を発生し, $r_j > u_j$ならば, ランダムサンプル $\eta^*_j$を採択し, $r_j < u_j$ならば, $\eta_j(t-1)$を保持する. サンプル数$T$が任意の数に達したとき, MCMCサンプリングを終了する. 

\end{enumerate}

\end{frame}

\begin{frame}{混合分布によるクラスタリング}
混合分布によるクラスタリングは, 対象となるデータが$m$個のコンポーネントからなる混合データであると仮定し, 混合分布における各コンポーネントのパラメータベクトル$\bm \eta$を推定する. 得られたパラメータベクトルとデータから, 各データがどのコンポーネントに所属しているかを確率的に表すことができる. 最も高い確率を示す, コンポーネントの番号をデータのラベルとすることで, 各データをクラスタリングすることができる. ここでの番号は, 各コンポーネントを表す, 名義尺度なので数値的な意味は存在しない. パラメトリックなクラスタリング手法では, 得られたパラメータベクトル$\bm \eta$の事後分布を用いて, 推定した混合分布の妥当性を検証することができる.	
\end{frame}

%%%%%%%%%%% シミュレーション %%%%%%%%%%%%%
\section{シュミレーション}
\begin{frame}{シュミレーション(1/2)}

\begin{itemize}
\item
得られた混合分布は, ともに元の分布の特徴をとらえている元になった.

\item
Mixture von Mises 分布では元のデータと同じく4つの元となる分布を推定したが, Mixture Projected Normal 分布においては3つの元となる分布により以下の結果が得られた.

\end{itemize}

\begin{table}[tbp]
\begin{center}
\caption{混合射影正規分布によるクラスタリング分析}
\label{cross}
\begin{tabular}{c|c|c c c c}
\hline
 &  & \multicolumn{4}{c}{Predict} \\ \hline
 &  & 1 & 2 & 3 & 4  \\ \hline 
 & 1 &  \textbf{301} & 27  & 94 & 78 \\ 
True
 & 2 & 2 & \textbf{473} & 1 & 24 \\
 & 3 & 140 & 14 & \textbf{274} &72 \\ 
 & 4 & 80 & 61 & 44 & \textbf{315} \\ 
\hline
\end{tabular}
\end{center}
\end{table}
\end{frame}

\begin{frame}{シュミレーション(2/2)}

\vspace{-1zh}
\begin{figure}[H]
\begin{tabular}{c}

\begin{minipage}{0.5\hsize}
\begin{center}
\includegraphics[clip,height= 45mm]{data/real.png}
\end{center}
\end{minipage}

\begin{minipage}{0.5\hsize}
\begin{center}
\includegraphics[clip,height= 45mm]{data/pred.png}
\end{center}
\end{minipage}
\end{tabular}
\caption{人工データの角度$\bm \theta = (\theta_1, \theta_2)^T$における散布図(左), 球面上の角度$\bm \theta = (\theta_1, \theta_2)^T$におけるクラスタリング結果(右)}
\end{figure}

\begin{itemize}
\item
複数のvon Mises 分布に従う乱数を発生させて, 1つのデータにまとめることで混合データを作成する.
\end{itemize}

\end{frame}

%%%%%%%%%  データ解析 %%%%%%%%%%%%
\section{データ解析}
\begin{frame}{解析データ}

本研究では, GroupLens による公開データセットMovieLensの一部分を用いた. MovieLensデータセットは映画評価サイト''movielens.com''において1997年9月から1998年4月までの7ヶ月間の間に集められた943人のユーザ, 1682個の映画についての10万個の評価値, 簡単なユーザ情報, コンテンツ情報から構成されている. 評価値において, $0$は見ていないことを表し, 評価については1から5までの5段階評価で数字が大きいほど高い評価である. 各ユーザは最低20個の映画に対する評価値を持っている. コンテンツ情報, ユーザ情報について表\ref{MovieLens1}にまとめる. また表\ref{MovieLens2}にMovieLensデータセットの一部を示す.


\begin{table}[bp]
\begin{center}
\caption{コンテンツ情報およびユーザ情報}   %キャプション
\label{MovieLens1}   %ラベル
\scalebox{0.7}{
\begin{tabular}{c l}
\hline
ジャンル & unknown, Action, Adventure, Animation, Children's, Comedy, \\
                 & Crime, Documentary, Drama, Fantasy, Film-Noir, Horror, Musical, \\
                 & Mystery, Romance, Sci-Fi, Thriller, War, Western \\
職業          & administrator, artist, doctor, educator, engineer, entertainment, executive, \\
                & healthcare, homemaker, lawyer, librarian, marketing, none, other, \\
                & programmer, retired, salesman, scientist, student, technician, writer \\
年齢 & 7歳 $\sim$ 73歳 \\
性別 & male, female\\ 
\hline
\end{tabular}
}
\end{center}
\end{table}

\end{frame}

\begin{frame}{コンポーネントの評価指標}

情報量基準(WAIC)を用いて, 混合分布のコンポーネントの選択を行う. AICやBICなどの情報量基準には, 事後分布が正規分布で近似されている必要があるなど, 様々な制約が存在するが, WAICは真の分布, 確率モデル, 事前分布がどのような場合でも用いることができる. WAICは式$(\ref{WAIC1})$で求められる.

\begin{eqnarray}
\label{WAIC1}
\mbox{WAIC} = T + \frac{V}{n}, 
\end{eqnarray}

\begin{table}[tbp]
\caption{WAICによるコンポーネントの選択結果}
\label{WAIC2}
\begin{center}
\scalebox{0.5}{
\begin{tabular}{l | c c c c c c c c c c}
\hline
 & 1 & 2 & 3 & 4 & 5 & 6 & 7 & 8 & 9 & 10 \\ \hline 
$m=1$ & -1014.1 & -1237.0 & -1350.6 & -1009.5 & -1308.8 & -1227.8 & -1152.4 & -902.5 & -1322.6 & -1140.2 \\ 
$m=2$ & -1610.9 & -1675.0 & -1760.4 & -1552.8 & -1633.1 & -1529.6 & -1507.1 & -1366.1 & -1645.2 & -1693.3 \\ 
$m=3$ & -1846.8 & -1744.7 & -1728.9 & -1804.9 & -1741.0 & -1654.3 & -1705.2 & -1633.8 & -1789.3 & \textbf{-1872.1} \\ 
$m=4$ & \textbf{-1893.6} & \textbf{-1969.8} & \textbf{-1901.6} & \textbf{-1837.5} & \textbf{-1787.5} & \textbf{-1687.8} &\textbf{-1749.5} & \textbf{-1686.7} & \textbf{-1844.5} & -1802.6 \\ 
\hline
\end{tabular}
}
\end{center}
\end{table}

\end{frame}

\begin{frame}{データ解析(1/3)}

$9$個のデータにおいて, コンポーネントが$4$つのときWAICが最小になっていることがわかる. コンポーネントを$4$つと仮定して混合射影正規分布によるクラスタリングを行う. サンプリング数を$20000$回, バーンイン期間を$10000$回として, MCMCサンプリングによりパラメータを推定した. $4$つのコンポーネントにおける混合射影正規分布のパラメータを式(\ref{parameter})に示す. なお, $\hat {\bm w}$は各コンポーネントの混合比率を表し, パラメータの添え字は各コンポーネントの番号に対応している.

\footnotesize %式を小さくする
\begin{equation}
\label{parameter}
\begin{split}
&\hat {\bm w} = \begin{pmatrix} 0.26 \\ 0.51 \\ 0.20 \\ 0.03 \\ \end{pmatrix},\ 
\hat{\bm \mu}_1 = \begin{pmatrix} 0.44 \\ 2.85 \\ -1.67 \\ \end{pmatrix},\ 
\hat \Sigma_1 = \begin{pmatrix}  2.40 & 1.52 &  0.06 \\ 1.52 & 2.20 & -0.25 \\ 0.06 & -0.25 &1.00 \\ \end{pmatrix},\\ 
&\hat{\bm \mu}_2 = \begin{pmatrix} 0.14 \\ -0.50 \\ 1.09 \\ \end{pmatrix},\ 
\hat \Sigma_2 = \begin{pmatrix}   0.28  & 0.14 &  0.20 \\ 0.14 & 0.61 & 0.06 \\  0.20 & 0.06 &1.00 \\ \end{pmatrix},\ 
\hat{\bm \mu}_3 = \begin{pmatrix} -2.08  \\ -1.96 \\ -8.11 \\ \end{pmatrix},\\ 
&\hat \Sigma_3 = \begin{pmatrix}  1.86  & -0.18 &  -0.10 \\-0.18 & 3.89 & 1.60 \\  -0.10 & 1.60 & 1.00 \\ \end{pmatrix},\ 
\hat{\bm \mu}_4 = \begin{pmatrix} -12.29   \\ -2.48 \\ -4.84 \\ \end{pmatrix},\ 
\hat \Sigma_4 = \begin{pmatrix} 15.19 & 4.01 &  0.12 \\ 4.01 & 1.88 & 0.08 \\ 0.12 & 0.08 &1.00 \\ \end{pmatrix}.
\end{split}
\end{equation}
\normalsize
\end{frame}


\begin{frame}{データ解析(2/3)}

\vspace{-1zh}
\begin{figure}[H]
\begin{tabular}{c}

\begin{minipage}{0.5\hsize}
\begin{center}
\includegraphics[clip,height= 40mm]{data/cluster_3d.png}
\end{center}
\end{minipage}

\begin{minipage}{0.5\hsize}
\begin{center}
\includegraphics[clip,height= 40mm]{data/cluster_4.png}
\end{center}
\end{minipage}

\end{tabular}
\label{fig2}
\caption{混合分布による, 球面上におけるクラスタリング結果(左), 混合分布による, 球面上の角度$\bm \theta = (\theta_1, \theta_2)^T$におけるクラスタリング結果(右)}
\end{figure}
\end{frame}

\begin{frame}{データ解析(3/3)}
\begin{figure}[tbp]
\begin{center}
\includegraphics[clip,height= 60mm]{data/cluster_plot.png}
\end{center}
\caption{クラスターごとのジャンルへの評価値}
\label{clustergenre}
\end{figure}

\end{frame}
%%%%%%%%%%%%%% まとめと今後の課題 %%%%%%%%%%%%%%%%%
\section{まとめと今後の課題}
\begin{frame}{まとめと今後の課題}

\begin{itemize}

\item
本研究では計算量の問題で高次元データをあらかじめ$3$次元に圧縮し, クラスタリングを試みたが, クラスターごとの特徴を明確に抽出することができなかった. 一般化射影正規分布では任意次元の超球面において分布を生成することができるので, 計算量の問題を解決し, 次元を圧縮することなく超球面上でのクラスタリングに取り組みたい. 本研究では数値データを球面上に配置し, クラスタリングを行ったのが, 超球面上でのクラスタリングは主にテキストマイニングや画像データに用いられているので, それらのデータに対するパラメトリックなクラスタリングを行い, 従来手法との比較を行いたい. 

\end{itemize}

\end{frame}

\section{参考文献}
\begin{frame}{参考文献}

{\scriptsize
\begin{thebibliography}{99}
%\setlength{\itemsep}{-.5zw}
\beamertemplatetextbibitems %参考文献に番号を振る

\bibitem{GPN}
D.Hernandez-Stumpfhauser and F. J. Breidt and Mark J. van der Woerd. (2017). The General Projected Normal Distribution of Arbitrary Dimension: Modeling and Bayesian Inference. {\it Journal of Bayesian Analysis}, Vol. 12, No. 1, pp. 113-133.

\bibitem{PN1}
F. Wang and A. E. Gelfand. (2013). Directional data analysis under the general
projected normal distribution. {\it Statistical Methodology}, Vol.10, pp. 113-127.

\bibitem{PN2}
B. Presnell, S. P. Morrison, and R. C. Littell. (1998). Projected multivariate linear models for
directional data. {\it Journal of the American Statistical Association}, Vol. 443, pp. 1068-1077.

%\bibitem{mvonMF}
%Jalil Taghia and Zhanyu Ma and Arne Leijon. (2014). Bayesian Estimation of the von-Mises Fisher Mixture %Model with Variational Inference. {\it IEEE}, Vol.36, No. 9, pp. 1701-1715.

\bibitem{vonMF}
S. Gopal and Y. Yang. (2014). Von Mises-Fisher Clustering Models. {\it ICML}.

\bibitem{skmeans}
S. Dhillon and S. Modha. (2001). Concept Decompositions for Large Sparse Text Data Using
Clustering. {\it Machine Learning}, Vol. 42, pp. 143-175.

\bibitem{movie}
GroupLensホームページ(http://movielens.org/.) 

\end{thebibliography}
}

\end{frame}
\end{document}

%\vspace{-0.5cm}
%\begin{figure}[H]
% \begin{center}
 % \includegraphics[width=40mm]{data/BN4.png}
 %\end{center}
 %\caption{ベイジアンネットワークの例}
 %\label{naive}
%\end{figure}