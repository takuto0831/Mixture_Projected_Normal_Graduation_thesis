\documentclass[a4paper]{jarticle}
\setlength{\textwidth}{170mm}
\setlength{\textheight}{260mm}
\setlength{\oddsidemargin}{-5mm}
\setlength{\topmargin}{-25mm}
\usepackage[dvipdfmx]{graphicx}
\usepackage{here}
\usepackage{theorem}
\usepackage{amsmath}
\usepackage{amsfonts}
\usepackage{ascmac}
\usepackage{bm}
\usepackage{comment}
\usepackage{listings,jlisting}
\usepackage{url}
\usepackage[square]{natbib}

\newtheorem{theo}{定理}[section]
\newtheorem{defi}{定義}[section]
\newtheorem{lemm}{命題}[section]

\title{ベイジアンネットワークによる予測モデルの構築}   %タイトル
\author{小坪琢人}   %著者
\date{\today}   %日付

\makeatletter
\def\theequation{\thesection.\arabic{equation}}   %数式番号を(章.式)形式
\@addtoreset{equation}{section}
\def\thefigure{\thesection.\arabic{figure}}   %図番号を(章.図)形式
\@addtoreset{figure}{section}
\def\thetable{\thesection.\arabic{table}}   %表番号を(章.表)形式
\@addtoreset{table}{section}
\def\tr{\mathop{\operator@font tr}\nolimits}
\def\grad{\mathop{\operator@font grad}\nolimits}
\def\St{\mathop{\operator@font St}\nolimits}
\def\Hess{\mathop{\operator@font Hess}\nolimits}
\def\D{\mathop{\operator@font D}\nolimits}
\def\sym{\mathop{\operator@font sym}\nolimits}
\makeatother

\setlength\textheight{230mm}   %テキスト高さ
\setlength\textwidth{160mm}   %テキスト幅
\setlength{\oddsidemargin}{0mm}   %余白

\begin{document}
\bibliographystyle{jecon}

\maketitle   %タイトルを付ける
\setlength{\baselineskip}{20pt}   %行間幅
\pagenumbering{roman}   %目次ページ番号のスタイル
\tableofcontents   %目次を付ける
\listoffigures   %図目次を付ける
\listoftables   %表目次を付ける
\clearpage   %目次と本文を分ける
\pagenumbering{arabic}   %本文ページ番号のスタイル

%%%%%これ以下, 本文%%%%%

\section{はじめに}
%%% 2-3ページ  %%%%

コンピュータ, ソフトウェア, インターネット技術などの情報技術は目覚しく発達し, 人間の生活はインターネットに大きく依存している形である. またインターネット通販サイトAmazon.comに見られる, その人にとってのお薦め商品を自動的に選ぶリコメンデーション機能や, Googleで用いられている重要なページを上位にランク付けする PageRank などは人間の意思決定にも大きく影響している. こうした機能は本人さえも気付いていない趣向性を明らかにすることもあるが, ユーザーの選択肢が狭める可能性がある.

Amazon.com リコメンドを例にとって説明すると, 各ユーザーの購入した商品を記録し, 同様の商品購入パターンを持つ他のユーザーとグループ化する. そのグループ内であるユーザはまだ購入していないが, 他のユーザーが購入しているものを, リコメンドするという仕組みである. この手法は協調フィルタリングと呼ばれている. Amazon のように多くのユーザーを抱える企業では, 様々な特徴をもつユーザーも観測できるので, 精度の高いリコメンドを行うことが出来る.

コンピュータが自動的に人間に情報を与えてくれるのは便利であるが, 問題点もある. 自動化の問題点としては, 分析やシュミレーションの実態が見えにくくなるという点である. 機械がある種のブラックボックスとなり, 都合よく利用されてしまう可能性がある. 人間が生活していくうえで, 重要視されるのは効率性よりも人間らしい判断である. その中で機械による効率的な自動化は, 意思決定に不信感をもたらす危険性を持っている.

それらを解決する方法として, 可視化というものが存在する. グラフやプロットにより数値的根拠を明らかにすることで, ブラックボックス化を防ぐことができる. 一般に可視化は得られた結果の解釈の手助けとして用いられているが, モデル自体が視覚的に解釈できるものも存在する. 例としては決定木やグラフィカルモデルなどである. 本論文ではこのグラフィカルモデルについて取り上げる. 

%%% グラフィカルモデルの本読んでもう少し書くよ
%%% 因果関係のことどこにいれる??

グラフィカルモデルは以下のような特徴を持つ.\citep{Bishop2}

\begin{enumerate}

\item 確率モデルの構造を視覚化する簡単な方法を提供し, 新しいモデルの設計方針を決めるのに役立つ.

\item グラフの構造を調べることにより, 条件付独立性などのモデルの性質に関する知見が得られる.

\item 精巧なモデルにおいて推論や学習を実行するためには複雑な計算が必要となるが, これを数学的な表現で暗に伴うグラフ上の操作として表現することができる.

\end{enumerate}

グラフはリンク(link)によって接続されたノード(node)の集合からなる. 確率的グラフィカルモデルでは, 各ノードが確率変数を, リンクがこれらの変数間の確率的関係を表現する. 本論では有向グラフィカルとも呼ばれる, ベイジアンネットワーク(Bayesian network)を用いた分類モデルについて議論する.

本研究ではベイジアンネットワークの構造学習における問題点を改良し, 確率変数間の因果関係を正しく推論した上で, ベイジアンネットワークを構築する.

%%%%%%%%%%%%%%%%%%%%% 用いるモデルの説明 %%%%%%%%%%%%%%%%%%%%%%%%%%%%%%%%
\section{ネットワークモデリング}

\subsection{ベイジアンネットワーク}

ベイジアンネットワークは, 確率変数間の条件付依存関係を表した非循環型有向グラフ(DAG: Directed Acyclic Graph)で, 各種推論などに用いられる. 各ノードは確率変数を表し, 有向辺は変数間の直接の依存関係を表す. 一般には各変数は離散値を取るが, 閾値を定めて離散値に変換すれば連続値にも適用可能である. ベイジアンネットワークでは, リンクの先にあるノードを子ノード$(X_j)$, リンクの元にあるノードを親ノード$(X_i)$と呼ぶ. 親ノードが複数あるとき子ノード$X_j$の親ノードの集合を$Pa(X_j)$と書くことにする. $X_j$と$Pa(X_j)$の間の依存関係は次の条件付確率によって表される. ただし$Pa(X_j)$が空集合の時, $X_j$はベイジアンネットワークの始点であるので, 事前確率分布($P(X_j))$となる.

\begin{eqnarray}
\label{eq1}
P(X_j | Pa(X_j)) = \frac{P(X_j, Pa(X_j))}{P(Pa(X_j))} 
\end{eqnarray}

さらに$n$個の確率変数$X_1, \dots, X_n$のそれぞれを子ノードとして同様に考えると, 全ての確率変数の同時確率分布は式(\ref{eq2})のように表せる.

\begin{eqnarray}
\label{eq2}
P(X_1, \dots, X_n) = \prod_{j=1}^n P(X_j | Pa(X_j))
\end{eqnarray}

上記の式(\ref{eq1})に基づいて, 各ノードについて条件付確率を求めた表を条件付確率表(CPT)といい, 各行の和は必ず$1$になる. 任意の子ノードはそのノードの親ノードのいずれかに起因するので総和は$1$となる (親ノード集合に含まれていないノードを条件とした条件付確率は$0$となる). 条件付確率表の例を表(\ref{cpt})に示す. 

\begin{table}[H]
\begin{center}
\caption{条件付確率表(CPT)}   %キャプション
\label{cpt}   %ラベル
\begin{tabular}{|c||c|c|c|}   %{}で文字の揃え方を指定
\hline
 & $Pa(X_{j})=x_{1}$ & \dots & $Pa(X_{j})=x_{m}$
\\ \hline
$X_j=y_1$ & $p(y_1|Pa(X_j)=x_1)$ & \dots & $p(y_1|Pa(X_j)=x_m)$
\\ \hline
$\vdots$ & $\vdots$ & $\ddots$ & $\vdots$
\\ \hline
$X_j=y_n$ & $ p(y_n|Pa(X_j)=x_1)$ & \dots & $p(y_n|Pa(X_j)=x_m)$
\\ \hline
\end{tabular}
\end{center}
\end{table}

ベイジアンネットワークモデルでは親ノードに対応する変数の値が与えられたとき, 逐次的に各ノードの値も計算される. つまり親ノード以外のノードの値が与えられているとき, ベイズの定理を用いて, 尤もらしい親ノードの値を予測することができる. ベイジアンネットワークによる予測モデルとして, Naive Bayes と Tree Augmented Naive Bayes について説明する.

\subsubsection{Naive Bayes}

Naive Bayes 型のベイジアンネットワークは, 親ノード以外のノードがすべて葉であるような木構造である. ナイーブベイズモデルにおける重要な仮定は, 親ノード以外のノードは, 親ノードに対応する変数の値が与えられた条件の下で条件付独立であるということである. ナイーブベイズモデルの例を図(\ref{naive})に示す.

\begin{figure}[H]
 \begin{center}
  \includegraphics[width=80mm]{data/sample1.png}
 \end{center}
 \caption{ナイーブベイズの例}
 \label{naive}
\end{figure}

構造は常に同様なので, パラメータ学習のみ行う. 訓練データから親ノードと各ノードの値を用いて条件付き確率を求める. 予測を行う際には, 各変数に値を与えてそれぞれ親ノードの値の確率を求め, それらの確率の総和で親ノードの値を決定する.

\subsubsection{Tree Augmented Naive Bayes (TAN)}

TANは, 親ノード以外のノード間が条件付独立であるという Naive Bayes の制約を少し緩めたもので, 子ノード間にも以下に示す条件の下でリンクを与えることができる.\citep{Friedman}

\begin{itemize}
\item 木構造であること

\item 全ての点の入次数は1以下であること

\item $n$個の説明変数に対して、$n-1$本の有向辺が存在すること
\end{itemize}

TAN 型のベイジアンネットワークを用いた分類器は, 各変数間の依存関係をある程度反映できるため, グラフ構造から視覚的に情報が得られる. TANモデル の例を図(\ref{TAN})に示す.

\begin{figure}[H]
 \begin{center}
  \includegraphics[width=80mm]{data/sample2.png}
 \end{center}
 \caption{Tree Augmented Naive Bayes}
 \label{TAN}
\end{figure}


TAN 型のベイジアンネットワークは Naive Bayes 型のベイジアンネットワークを元に, 関係性の強い説明変数について相互情報量をもとにリンクを与える. その際, 元のクラスラベルの値を条件とした条件付相互情報量を用いる. ここでは条件付相互情報量をTAN 構造の仮定の下で最大化する.

まず, エントロピーについて説明する. エントロピーは式(\ref{entropy})のように表され, 不確かさ(乱雑さ)の指標とされる. $p(x)$は各変数の取りうる値の出現確率を表している. 全ての確率が等しいとき($1/N)$, エントロピーは最大となり, $\log N$となる. あるひとつの値の確率が$1$となり, 他の全ての確率が$0$となるとき, エントロピーは最小となり, $0$である. 同様に結合エントロピー(式(\ref{jointentropy})), 条件付エントロピー, (式(\ref{conditionalentropy}))についても示しておく.

\begin{eqnarray}
\label{entropy}
H(X) &=& - \int_x p(x) \log p(x) 
\end{eqnarray}

\begin{eqnarray}
\label{jointentropy}
H(X, Y) &=& - \int_x \int_y p(x) p(y) \log p(x) \log p(y) \\
\label{conditionalentropy}
H(X| Y) &=& - \int_x \int_y p(x, y) \log p(x| y) dx dy \nonumber \\
           &=& - \int_x \int_y p(x, y) \log p(x, y) dx dy + \int_x \int_y p(x, y) \log p(x) dx dy \nonumber \\
           &=& H(X, Y) - H(Y)
\end{eqnarray}


次に KL情報量(相対エントロピー)\citep{Meyer}について説明する. KL情報量は式(\ref{KL}) に示したように, 確率分布の近似に用いられる. KL情報量が小さいほど二つの分布が近似していることを表し, $p(x) = q(x)$ となるときKL情報量は$0$となる.

\begin{eqnarray}
\label{KL}
KL(p(x) | q(x)) &=& - \int p(x) \log q(x) dx - \left(- \int p(x) \log p(x) dx \right) \nonumber \\ 
                    &=& - \int p(x) \log \frac{q(x)}{p(x)} dx
\end{eqnarray}

相互情報量と条件付相互情報量について以下のように定義する.

\begin{eqnarray*}
\mbox{相互情報量} &=& I(X, Y) \\
\mbox{条件付相互情報量} &=& I(X, Y | C)
\end{eqnarray*}

ここで相互情報量と条件付相互情報量をKL情報量の式で表すと以下のように表せる. 相互情報量は$X$と$Y$の依存度を表している. 式(\ref{MI})より $X$と$Y$が独立のとき$p(x, y) = p(x) p(y)$となり, KL情報量の定義より $I(X, Y) = 0$となる. すなわち依存関係がある場合に相互情報量は大きくなる. 条件付相互情報量についても同様である. 

\begin{eqnarray}
\label{MI}
I(X, Y) &=& KL(p(x, y) | p(x) q(x)) \nonumber \\
          &=& - \int \int p(x, y) \log \frac{p(x, y)}{p(x) q(x)} dx dy 
\end{eqnarray}

\begin{eqnarray}
\label{CMI}
I(X, Y | Z) &=&  KL(p(x, y| z) | p(x|z) q(x|z)) \nonumber \\
              &=& - \int \int \int p(x, y, z) \log \frac{p(x, y|z)}{p(x|z) q(x|z)} dx dy dz \nonumber \\
               &=& - \int \int \int p(x, y, z) \log p(x, y, z) dx dy dz  \nonumber \\
                &&+ \int \int \int p(x, y, z) \log p(x| z) dx dy dz 
\end{eqnarray}

相互情報量と条件付相互情報量は, エントロピーの定義式(式(\ref{entropy}), 式(\ref{jointentropy}), 式(\ref{conditionalentropy}))を用いて置き換えることができる.  

\begin{eqnarray}
\label{MIentropy}
I(X, Y) &=& H(X) - H(X|Y) \nonumber \\
         &=& H(X) - H(X, Y) + H(Y)
\end{eqnarray}

\begin{eqnarray}
\label{CMIentropy}
I(X, Y | Z) &=& H(X| Z) - H(X| Y, Z) \nonumber \\
               &=& H(X, Z) + H(Y, Z) - H(X, Y, Z) - H(Z)
\end{eqnarray}

TAN 型のベイジアンネットワークでは式(\ref{CMIentropy})に示した, 条件付相互情報量の式を元に親ノード以外のノード間にリンクを与えている. パラメータ学習はナイーブベイズモデルと同様に,  訓練データから親ノードと各ノードの値を用いて条件付き確率を求める. 予測を行う際には, 各変数に値を与えてそれぞれ親ノードの値の確率を求め, それらの確率の総和で親ノードの値を決定する.

\subsection{統計的因果推論}

統計的因果推論は, 因果関係をデータから推測する方法論である.\citep{清水昌平} 

%%%%%%%%%%%%%%% ここまでモデル説明 %%%%%%%%%%%%%%%%%%%%%%%%%%

\section{データ解析}

\subsection{解析データ}

本研究では, ある美容院の店舗マスタ, 担当者マスタ, 商品マスタ, 顧客マスタ, 会計履歴, 会計明細を用いて顧客のクラスタリングを行う. 
以下にデータの詳細を示す. 

\begin{table}[H]
\begin{center}
\caption{解析データ}   %キャプション
\label{hairdata}   %ラベル
\begin{tabular}{c l}
\hline
店舗マスタ & 店舗ID, 店舗名, 緯度, 経度, 階数 \\

担当者マスタ & 担当者ID, 担当者所属店舗 \\

商品マスタ & 会計明細販売商品ID, 商品種別,  第1カテゴリID, 第1カテゴリ, 第2カテゴリID, \\
               &  第2カテゴリ, 商品名, 商品略称, 本部設定税抜価格, 本部設定税込価格, 施術分数, \\
               & 販売開始日, 販売終了日, 予約カテゴリー, 予約商品名\\

顧客マスタ & 顧客ID, 初回来店年, 初回来店店舗ID, 初回来店店舗ID, 郵便番号, \\
              & DM送信可否, 性別, 誕生年代, 備考 \\

会計履歴 & 会計ID, 販売店舗ID, 会計日, 会計時刻, 顧客ID, 会計税込売上, 会計消費税,取引種別, \\
             &  POS入力担当者ID, 会計主担当者ID, 会計指名区分, DV商品券金額, DV商品券枚数,\\
             &  現金, クレジット, 電子マネー, 売掛金, 他社商品券金額, 他社商品券枚数, 商品券利用枚数, \\
             &  利用商品券ID,ポイント利用額, ポイント利用区分, ポイント付与額, \\
            &  ポイント残高, 累積来店回数, 曜日\\

会計明細 & 会計明細ID, 会計ID, 明細商品種別, 会計明細税込定価売上, 会計明細割引額, \\
              & 会計明細消費税, 会計明細販売商品ID, 会計明細販売担当者ID, 会計明細指名区分 \\
\hline
\end{tabular}
\end{center}
\end{table}

\subsection{評価基準}

性能評価の基準として, 予測されたレーティング$\hat{r}_{i, u}$と真のレーティング$r_{i, u}$との平均二乗誤差(Mean Squared Error: MSE):

\begin{eqnarray}
\label{MSE}
\frac{1}{|\mathcal W|} = \sum_{(i, u) \in \mathcal W} (\hat{r}_{i, u} - r_{i, u})^2
\end{eqnarray}

を用いた. ここで, $\mathcal W$は評価に用いるデータに含まれるレーティングのインデックス集合であり, $|\mathcal W|$は$\mathcal W$に含まれる要素の数, すなわちレーティングの総数を表す. MSEが小さいほど予測の精度が高いことになる.

\subsection{実験結果}

第2章で示した, Naive Bayes 型, TAN 型のベイジアンネットワークモデルを用いて, 予測を行う. TAN 型のベイジアンネットワークについては式(\ref{CMIentropy})を用いて, 関係性の強い変数を決定する. 変数として, age, gender, occupation, item.No を選んだ場合の条件付相互情報量を表(\ref{CMIsample})に示す.

\begin{table}[H]
\begin{center}
\label{CMIsample}
\caption{条件付相互情報量}
\begin{tabular}{|c||c|c|c|c|} \hline  
& age & gender & occupation & item.No \\ \hline \hline
age & 3.666 & 0.0618 & \bf{0.873} & \bf{0.981} \\
gender &  & 0.573 & \bf{0.0988} & 0.0644 \\
occupation &  &  & 2.583 & 0.517 \\
item.No &  &  &  & 6.564 \\ \hline
\end{tabular}
\end{center}
\end{table}

実際にリンクを与える部分を太字で強調した. 表からわかるように (ocupation, item.No) 間のほうが値が大きいがここに線を引くと閉路ができてしまうのでその次に大きい値を用いていることがわかる.

これらの手順により相互情報量をもとに無向辺を与える. ここで相互情報量をもとに有向辺を与えることができないことに注意する. 有向辺を決定するのはTAN 型のベイジアンネットワークにおける条件を用いてベイジアンネットワークを構成する.

前項で定めたグラフモデルをもとに, 全変数の取りうる値について条件付確率を求める. ここでも一例を表(\ref{CPTsample})に示す. これらの条件付確率表(CPT)をもとにテストデータについて最も確率の高いレイティングを出力とする.

\begin{table}[H]
\begin{center}
\label{CPTsample}
\caption{条件付確率表の例 (occupatin = executive)}
\begin{tabular}{|c||c|c|c|c|c|} \hline  
gender/rating & 1 & 2 & 3 & 4 & 5 \\ \hline \hline
F & 0.113 & 0.089 & 0.242 & 0.121 & 0.147 \\
M & 0.886 & 0.910 & 0.757 & 0.878 & 0.852 \\ \hline
\end{tabular}
\end{center}
\end{table}

\subsubsection{性能評価}

第2章で示した, Naive Bayes 型, TAN 型のベイジアンネットワークモデルを用いて, 予測を行う. 性能評価はMSEを用いた. 結果を表(\ref{modelmse})に示す.

\begin{table}[H]
\begin{center}
\caption{各手法における性能評価}   %キャプション
\label{modelmse}   %ラベル
\begin{tabular}{l c}
\hline
                            & MSE \\ \hline
Matrix Factorization &  0.8832 \\ \hline
Naive Bayes model  & 1.3027 \\ \hline
TAN model             & 2.0558 \\ \hline
\end{tabular}
\end{center}
\end{table}

\section{まとめと今後の課題}
%%%%%そもそもまだ勝ってません

AdaBoost, バギングとの組み合わせを実装する. 以前読んだ論文でも Boosting の手法を取り入れたベイジアンネットワークを構築していたので, その論文を参考にする. また, 性能評価の方法が適切であるかを検討する. 

\section{謝辞}
%%%%%%%%書きましょう
本研究を行うにあたり, ご指導を頂いた塩濱教授に感謝しています. またお忙しい中, 学部生の質問に対応して頂いた院生の皆様にも感謝しています.

\newpage
\addcontentsline{toc}{section}{参考文献} %目次に参考文献を入れる

%必要になる
%\newpage
%\section{付録}

%参考文献を引用する際に必要なコマンド
\bibliographystyle{jplain}
\bibliography{bunken}

\end{document}