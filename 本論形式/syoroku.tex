\documentclass[a4j,11pt]{jarticle}
\usepackage[dviout]{graphicx}
\usepackage{amsmath}
\usepackage{amssymb}
\usepackage{ascmac}
%\usepackage{epsbox}
\usepackage{float}
\usepackage{graphics}
\usepackage{here}
\usepackage{lscape}
\usepackage{latexsym}
\usepackage{pifont}
\usepackage{wrapfig}
\usepackage{type1cm}
\usepackage{algorithm}
\usepackage{algorithmic}
\usepackage{txfonts}
\usepackage{bm}
\usepackage[dviout]{graphicx}
\usepackage{comment}
\usepackage{url}
\usepackage{natbib}

%\bibliographystyle{apalike} %いらない
%\setlength{\voffset}{-25.4mm}
\setlength{\topmargin}{-17.5mm}   %トップとヘッダの間隔
%\setlength{\headheight}{20mm}   %ヘッダの高さ
%\setlength{\headsep}{0mm}   %ヘッダとテキストの間隔
\setlength{\textwidth}{45zw}   %テキストの幅
\setlength{\hoffset}{-10mm}
\setlength{\textheight}{45\baselineskip}   %テキストの高さ
%\addtolength{\textheight}{\topskip}
%\setlength{\footskip}{0mm}
%\setlength{\oddsidemargin}{21.5mm}   %サイドとテキストの間隔(奇数ページ)
%\setlength{\evensidemargin}{21.5mm}   %サイドとテキストの間隔(偶数ページ)
\pagestyle{empty}   %ページ番号なし
\newcommand{\g}[1]{\boldsymbol{#1}}
\newcommand{\lw}[1]{\smash{\lower2.0ex\hbox{#1}}}
\renewcommand{\baselinestretch}{1.0}

\makeatletter
\def\theequation{\arabic{equation}}   %数式番号を(章.式)形式
\@addtoreset{equation}{section}
%\def\thefigure{\thesection.\arabic{figure}}   %図番号を(章.図)形式
%\@addtoreset{figure}{section}
%\def\thetable{\thesection.\arabic{table}}   %表番号を(章.表)形式
%\@addtoreset{table}{section}
\def\tr{\mathop{\operator@font tr}\nolimits}
\def\grad{\mathop{\operator@font grad}\nolimits}
\def\St{\mathop{\operator@font St}\nolimits}
\def\Hess{\mathop{\operator@font Hess}\nolimits}
\def\D{\mathop{\operator@font D}\nolimits}
\def\sym{\mathop{\operator@font sym}\nolimits}
\def\s.t.{\mathop{\operator@font s.t.}\nolimits}
\def\diag{\mathop{\operator@font diag}\nolimits}
\def\section{\@startsection{section}{1}{\z@}
   {0.8\Cvs \@plus.5\Cdp \@minus.2\Cdp}
   {0.2\Cvs \@plus.3\Cdp}
   {\normalfont \Large \bfseries}}
\makeatother
\makeatletter
\def\subsection{\@startsection{subsection}{1}{\z@}
   {0.8\Cvs \@plus.5\Cdp \@minus.2\Cdp}
   {0.2\Cvs \@plus.3\Cdp}
   {\normalfont \normalsize \bfseries}}
\makeatother
\makeatletter
\newcommand{\figcaption}[1]{\def\@captype{figure}\caption{#1}}
\newcommand{\tblcaption}[1]{\def\@captype{table}\caption{#1}}
\makeatother
\begin{document}

\bibliographystyle{jecon}
\begin{center}
{\Large \textbf{ベイジアンネットワークによる予測モデル}}
\end{center}
\begin{flushright}
小坪 琢人(塩濱 敬之准教授,佐藤 寛之助教)
\end{flushright}
\vspace{-3zw}

%%%%%これ以下, 本文%%%%%

\section{はじめに}
\vspace{-0.6zh}
現在, 様々な分野で機械学習やディープラーニングが用いられている. その中で自動化というものが大きなテーマであり, 「今後自動化される仕事」のようなニュースも増えている. しかし, マシンの進化により失われる仕事があるのなら, 生まれる新しい仕事も当然存在する. これから生まれる仕事, 必要になる仕事を考えるためには, まず機械学習やディープラーニングでどのようなことができるのかを知る必要がある. 

自動化の問題点としては, 分析やシュミレーションの実態が見えにくくなるという点である. 機械がある種のブラックボックスとなり, 都合よく利用されてしまう可能性がある. 人間が生活していくうえで, 重要視されるのは効率性よりも人間らしい判断である. その中で機械による効率的な自動化は, 意思決定に不信感をもたらす危険性を持っている.

それらを解決する方法として, 可視化というものが存在する. グラフやプロットにより数値的根拠を明らかにすることで, ブラックボックス化を防ぐことができる. 一般に可視化は得られた結果の解釈の手助けとして用いられているが, モデル自体が視覚的に解釈できるものも存在する. 例としては決定木やグラフィカルモデルなどである. 本論文ではこのグラフィカルモデルについて取り上げる. 

%%% グラフィカルモデルの本読んでもう少し書くよ
%%% 因果関係のことどこにいれる??

グラフィカルモデルは以下のような特徴を持つ.\citet{Bishop2}

\begin{enumerate}

\item 確率モデルの構造を視覚化する簡単な方法を提供し, 新しいモデルの設計方針を決めるのに役立つ.

\item グラフの構造を調べることにより, 条件付独立性などのモデルの性質に関する知見が得られる.

\item 精巧なモデルにおいて推論や学習を実行するためには複雑な計算が必要となるが, これを数学的な表現で暗に伴うグラフ上の操作として表現することができる.

\end{enumerate}

グラフはリンク(link)によって接続されたノード(node)の集合からなる. 確率的グラフィカルモデルでは, 各ノードが確率変数を, リンクがこれらの変数間の確率的関係を表現する. 本論では有向グラフィカルとも呼ばれる, ベイジアンネットワーク(Bayesian network)を用いた分類モデルについて議論する.

%%%%%%%%%%%%%%%%%%%%%%%%%%%%%%%%%%%%%%%%%%%%%%%%%%%%%%%%%%%%%%%%%%%%%%%%%%%%%%%%%%%%%%%%%%%%%%%%%%%%%%%%%%%%%%%%%%

\section{理論}

\subsection{ベイジアンネットワーク}

ベイジアンネットワークは, 確率変数間の条件付依存関係を表した非循環型有向グラフ(DAG: Directed Acyclic Graph)で, 各種推論などに用いられる. 各ノードは確率変数を表し, 有向辺は変数間の直接の依存関係を表す. 一般には各変数は離散値を取るが, 閾値を定めて離散値に変換すれば連続値にも適用可能である. ベイジアンネットワークでは, リンクの先にあるノードを子ノード$(X_j)$, リンクの元にあるノードを親ノード$(X_i)$と呼ぶ. 親ノードが複数あるとき子ノード$X_j$の親ノードの集合を$Pa(X_j)$と書くことにする. $X_j$と$Pa(X_j)$の間の依存関係は次の条件付確率によって表される. ただし$Pa(X_j)$が空集合の時, $X_j$はベイジアンネットワークの始点であるので, 事前確率分布($P(X_j))$となる.

\begin{eqnarray}
\label{eq1}
P(X_j | Pa(X_j)) = \frac{P(X_j, Pa(X_j))}{P(Pa(X_j))} 
\end{eqnarray}

さらに$n$個の確率変数$X_1, \dots, X_n$のそれぞれを子ノードとして同様に考えると, 全ての確率変数の同時確率分布は式(\ref{eq2})のように表せる.

\begin{eqnarray}
\label{eq2}
P(X_1, \dots, X_n) = \prod_{j=1}^n P(X_j | Pa(X_j))
\end{eqnarray}

上記の式(\ref{eq1})に基づいて, 各ノードについて条件付確率を求めた表を条件付確率表(CPT)といい, 各行の和は必ず$1$になる. 任意の子ノードはそのノードの親ノードのいずれかに起因するので総和は$1$となる (親ノード集合に含まれていないノードを条件とした条件付確率は$0$となる). 条件付確率表の例を表(\ref{cpt})に示す. 

\begin{table}[H]
\begin{center}
\caption{条件付確率表(CPT)}   %キャプション
\label{cpt}   %ラベル
\begin{tabular}{|c||c|c|c|}   %{}で文字の揃え方を指定
\hline
 & $Pa(X_{j})=x_{1}$ & \dots & $Pa(X_{j})=x_{m}$
\\ \hline
$X_j=y_1$ & $p(y_1|Pa(X_j)=x_1)$ & \dots & $p(y_1|Pa(X_j)=x_m)$
\\ \hline
$\vdots$ & $\vdots$ & $\ddots$ & $\vdots$
\\ \hline
$X_j=y_n$ & $ p(y_n|Pa(X_j)=x_1)$ & \dots & $p(y_n|Pa(X_j)=x_m)$
\\ \hline
\end{tabular}
\end{center}
\end{table}

\section{数値実験}

本研究では, GroupLensプロジェクトによる公開データセット MovieLens\citet{MovieLens} の一部分を用いた. MovieLensデータセットは映画評価サイト''movielens.com''において1997年9月から1998年4月までの7ヶ月間の間に集められた943人のユーザ, 1682個の映画についての10万個のレーティングデータ, 簡単なユーザ情報, コンテンツ情報から構成されている. レーティングデータは, 1から5までの5段階評価で数字が大きいほど高い評価である. 各ユーザは最低20個のレーティングを持っている. ユーザ情報, コンテンツ情報について表(\ref{MovieLens})にまとめる. 

\begin{table}[H]
\begin{center}
\caption{ユーザ情報およびコンテンツ情報}   %キャプション
\label{MovieLens}   %ラベル
\begin{tabular}{c l}
\hline
映画ジャンル & unknown, Action, Adventure, Animation, Children's, Comedy, Crime, \\
                 & Documentary, Drama, Fantasy, Film-Noir, Horror, Musical, Mystery, \\
                 & Romance, Sci-Fi, Thriller, War, Western \\
職業          & administrator, artist, doctor, educator, engineer, entertainment, executive, \\
                & healthcare, homemaker, lawyer, librarian, marketing, none, other, programmer, \\
                & retired, salesman, scientist, student, technician, writer \\
年齢 & 5歳ごとに分割 or 10歳ごとに分割 \\
性別 & male, female\\ 
\hline
\end{tabular}
\end{center}
\end{table}

\section{まとめ}
%%%%%%%%%%%%%%%%%%%%%%%%%%%%%%%%%%%%%%%%%%%%%%%%%%%%%%%%%%%%%%%%%%%%%%%%%%%%%%%%%%%%%%%%%%%%%%%%%%%%%%%%%%%%%%%%%%

\newpage
\addcontentsline{toc}{section}{参考文献} %目次に参考文献を入れる

%必要になる
%\newpage
%\section{付録}

%参考文献を引用する際に必要なコマンド
\bibliographystyle{jplain}
\bibliography{bunken}

\end{document}